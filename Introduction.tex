%% licence RAFRAB(*)
%%
%% RAFRAB : rien à f**tre, rien à battre, fais ce qu'il te plait dans la vie
%%


\documentclass[french]{beamer}
\usepackage{babel}
\usepackage[utf8]{inputenc}
\usetheme{Warsaw}

\title{Introduction à l'administration \\ \textit{d'un système GNU/Linux}}
\author{Ndimby Andriantsoavina}
\institute{Agence universitaire de la Francophonie}

\begin{document}

\begin{frame}
\titlepage
\end{frame}

\section{Généralités}
\frame{\tableofcontents[current]}

%\begin{frame}
%\textbf{Linux est basé sur UNIX}
%    \begin{itemize}
%    \item philosophie
%    \item commandes
%    \item standard et conventions UNIX
%    \end{itemize}
%\end{frame}


\begin{frame}
\begin{center}
\textbf{Rôles de l'administrateur}
\end{center}
\begin{itemize}
\item Créer, modifier, supprimer un utilisateur
\item Gérer les fichiers et les disques
\item Surveiller l'espace disque
\item Organiser les sauvegardes
\item Ajoutes des périphériques
\item Améliorer les performances du système
\item Intégrer de nouveaux services
\item Veiller à la sécurité du système
\item Paramétrer le démarrage et l'arrêt du système
\item Suivre les logs du système
\end{itemize}
\end{frame}

\begin{frame}
\textbf{LINUX} :
    \begin{itemize}
    \item développé par Linus Torvalds en 1991
    \item au sens strict c'est le noyau ou kernel linux
    \item interface entre le matériel et les applications, les droits d'accès
    \item gère la mémoire, les périphériques 
    \end{itemize}
\end{frame}

\begin{frame}
Le système d'exploitation \textbf{GNU/Linux}:
    \begin{itemize}
    \item \textbf{Multi-tâches}: execution 'simultanée' de plusieurs programmes
    \item \textbf{Multi-utilisateurs}: chaque utilisateur à un login/mdp
    \item \textbf{libre} : liberté d'éxecuter, de copier, de redistribuer, d'étudier, de
    modifier et d'améliorer
    \item \textbf{multi-processeurs}: plusieurs processeurs
    \item \textbf{multi-plateformes}: PC, SPARC, système embarqué, téléphones
    mobiles
    \end{itemize}
\end{frame}

\begin{frame}
\textbf{Distributions Linux} \\
ensemble cohérent de logiciels assemblé autour du système d'exploitation
\texttt{GNU} et du noyau \texttt{Linux}
    \begin{itemize}
    \item Mandriva 
    \item RedHat
    \item Debian
    \item Ubuntu
    \item Fedora
    \end{itemize}
\end{frame}

\begin{frame}
Pourquoi \textit{"on"} utilise \textbf{Debian}?

\begin{itemize}
\item noyau Linux et outils GNU : Debian GNU/Linux
\item système de gestion de paquets avec plus de \textbf{20000} paquets... avec suivi de sécurité
\item \textbf{stabilité}: plus de 1000 développeurs officiels (et 2000 contributeurs)
\item disponible pour \textbf{11 architectures} 
\item Pour plus d'infos \url{http://www.debian.org/}
\end{itemize}
\end{frame}

\begin{frame}
Les versions de Debian GNU/Linux
\begin{itemize}
\item \textbf{stable} contient la dernière distribution officiellement sortie de Debian (squeeze)
\item \textbf{testing} contient les paquets qui n'ont pas encore été acceptés dans la distribution stable, mais qui sont en attente de l'être (wheezy)
\item \textbf{unstable}  est celle sur laquelle les activités de développement se déroulent (Sid)
\end{itemize}
\end{frame}

\begin{frame}
Un peu de lecture avant l'installation...
\begin{itemize}
\item Manuel d'installation : \url{http://www.debian.org/releases/stable/installmanual}
\item Installation via le réseau : \url{http://www.debian.org/distrib/netinst}
\end{itemize}
\textbf{TP:} Installation de Débian par le réseau
\end{frame} 

\section{Système de fichiers}
\frame{\tableofcontents[current]}

\begin{frame}
\begin{center}
Système de fichiers: façon d'organiser et de stocker une arborescence sur
un support (disque, disquette, cd ...)
\end{center}
    \begin{itemize}
    \item \textbf{MS-DOS} : FAT16/32 (File allocation Table)
    \item \textbf{Windows} : FAT16, FAT32, VFAT, NTFS, exFAT(extended FAT)
    \item \textbf{Linux} : ext2/3/4 (extended File System), HFS, ReiserFS, XFS, swap ...
    \end{itemize}
\end{frame}

\begin{frame}
L'arborescence système :
    \begin{itemize}
    \item \textbf{/} : répertoire racine doit contenir tout ce qui permet au système de démarrer
    \item \textbf{/bin}, \textbf{/sbin} : contenant les exécutables de base (ex: cp, mv, ls)
    \item \textbf{/lib} : bibliothèques associées 
    \item \textbf{/etc} : contient les fichiers de configuration du système
    \item \textbf{/usr} (\textit{unix system ressources}) : ressources non nécessaires au démarrage contenant des programmes, documentation
    \end{itemize}
 \end{frame}
 
 \begin{frame}
	\begin{itemize}
    \item \textbf{/var} : répertoires des données variables (ex: logs)
    \item \textbf{/home} : répertoires des utilisateurs
    \item \textbf{/mnt} et \textbf{/media} : points de montages    
    \item \textbf{/tmp} : données temporaires
    \item \textbf{/proc} et \textbf{/sys} : systèmes de fichiers virtuels
    \item \textbf{/dev} : fichiers spéciaux nommés \texttt{device} correspondants à des périphériques
    \end{itemize}
\end{frame}


\section{Les commandes de bases} 
\frame{\tableofcontents[current]}

\begin{frame}
\textbf{Shell}: interpréteur de commandes, interface utilisateur et système
d'exploitation
    \begin{itemize}
    \item accès à la machine: login/motdepasse 
    \item lancement de commandes à partir d'un interpréteur \textit{Shell}
    \item syntaxe: \textit{prompt\# [commande] [option(s)] [arguments]}
    \item exemple: \textit{ls -l /home}
    \end{itemize}
\end{frame}

\begin{frame}
\textbf{Opération sur les répertoires}
    \begin{itemize}
    \item \textbf{ls} : lister les répertoires (syntax: \texttt{ls [-options][repertoires]})
    \item \textbf{mkdir} : création de répertoire (syntax:\texttt{mkdir [-options][nom du répertoire]})
    \item \textbf{rmdir} : suppression de répertoire (syntax:\texttt{rmdir [-options][nom du répertoire]})
    \item \textbf{pwd} : afficher le répertoire courant
    \item \textbf{cd} : changer de répertoire (syntax:\texttt{cd [nom du répertoire]})
    \item \textbf{du} : espace disque utilisé par le répertoire (syntax:\texttt{du [option][nom du répertoire]})
    \item \textbf{df} : espace disque occupé (ou libre) sur chaque partition (syntax: \texttt{df [-options][fichier]})
    \end{itemize}
\end{frame}

\begin{frame}
\textbf{Opération sur les fichiers}
    \begin{itemize}
    \item \textbf{cp} : copier les fichiers ou les répertoires (syntax: \texttt{cp [-options]<source><destination>})
    \item \textbf{mv} : déplace ou renommer les fichiers ou les répertoires (syntax: \texttt{mv [-options]<source><destination>})
    \item \textbf{rm} : effacer un fichier (syntax: \texttt{rm [-options][fichier]})
    \item \textbf{find} : chercher un fichier (syntax: \texttt{find [-répertoire de départ][option][action]})
    \item \textbf{which} : Montrer le chemin d'accès à une commande (syntax: \texttt{which [nom de pgm]})
    \item \textbf{less ou more} : afficher à l'écran le contenu d'un fichier text (syntax: \texttt{less [-options]<nom de fichier>})
    \item \textbf{whereis et locate}: permet de retrouver les emplacements des fichiers
    \end{itemize}
\end{frame}

\begin{frame}
\textbf{Les filtres}: commandes qui, à partir d'un flux d'entrées donné,
effectuent des traitements avant d'afficher un résultat en sortie.
    \begin{itemize}
    \item \textbf{cat} : concaténatation d'un fichier et afficher sur la sortie
    standard. \\
    \item \textbf{cut}: afficher suivant les options certains champs d'un
    fichier \\
    \textit{cut -d : -f1 /etc/passwd}
    \item \textbf{grep} : cherche une expression dans un fichier \\
    \textit{grep false /etc/passwd} \\
    \textit{grep -v root /etc/passwd} \\
    \item \textbf{sort}: tri ordre croissant
    \item \textbf{join}: effectue des jointures des lignes de deux fichiers
    différents dans un même champs. 
    \item \textbf{paste}: pour fusionner deux fichiers différents en prenant
    chaque ligne de chaque fichier pour former une nouvelle ligne
    \end{itemize}
\end{frame}


\begin{frame}
\textbf{Redirections} \\
    \begin{itemize}
    \item Entrée standard: descripteur \textbf{0}
    \item Sortie standard: descripteur \textbf{1}
    \item Sortie d'erreur standard: descripteur \textbf{2}
    \item \texttt{echo 'Bonjour Dakar' > hello.txt}
    Sortie standard et sortie d'erreur redirigé vers un fichier
    \item \texttt{tail /etc/passwd > fichiersortie 2>\&1 }
    \item \texttt{tail /etc/passwd >\& fichiersortie 2>1 }
    \end{itemize}
\end{frame}


\begin{frame}
\textbf{Tubes}
    \begin{itemize}
    \item \textbf{pipe} permet que la sortie d'une commande devient l'entrée
    d'une autre \\
    ex: \texttt{sort /etc/passwd | head -6}
    \item \textbf{tee}: recopie la sortie dans un fichier \\et affiche en même
    temps le résultat sur la sortie standard \\
    ex: \texttt{ls -l /etc | tee fichier | wc  -l}
    \item \textbf{xargs} passe en argument d'une commande les flux reçus en
    entrée\\
    ex: \texttt{cat lefichier | xargs ls}

    \end{itemize}
\end{frame}




\begin{frame}
Organisation des partitions sous Linux
\begin{itemize}
\item assurer une meilleure résistance du système aux crash disques
\item face aux attaques de tout types \textit{(ex: saturation de /home/abdou)}
\item failles de sécurité \textit{(ex: \texttt{/tmp} accessible en écriture à tous les utilisateurs et programmes )}
\end{itemize}
\end{frame}

\section{Gestion des périphériques}
\frame{\tableofcontents[current]}

\begin{frame}
\begin{center}
\textbf{Gestion des périphèriques}
\end{center}
\begin{itemize}
\item Périphériques, disques durs, et partitions gérés sous forme de fichier dans /dev
\item Chaque périphérique correspond un point de montage
\item \textbf{/dev/hda} : disque dur maître du cannal IDE 1
\item \texttt{/dev/hda1} : Première partition du DD IDE 1
\item \texttt{/dev/hda2} : Deuxième partition du DD IDE 1
\item \textbf{/dev/hdb} : disque dur esclave du cannal IDE1
\item \texttt{/dev/sda} : disque dur maitre SATA1
\item \textbf{Partitions primaires sont numérotées de 1-4}
\item \textbf{Partitions logiques sont numérotées à partir de 5-20}
\end{itemize}
\end{frame}

\begin{frame}
Les périphériques spéciaux
	\begin{itemize}
	\item \textbf{/dev/null} : tout ce qui est envoyé à ce périphérique est détruit
	\item \textbf{/dev/zero} : on peut lire une infinité de zéros depuis ce périphérique
	\item \textbf{/dev/random} : on peut lire des nombre aléatoires depuis ce périphérique
	\end{itemize}
\end{frame}




\begin{frame}
\textbf{Utiliser fdisk}
\begin{itemize}
\item \textbf{fdisk} permet de créer, effacer, et changer les partitions sur
un disque
\item ex: \texttt{fdisk /dev/hda}
\item \textbf{p}: afficher la table de partition
\item \textbf{n}: crée une nouvelle partition (primaire ou logique)
\item spécifier la taille de la partition
\item \textbf{w}: changer la table de partition
\end{itemize}
\end{frame}

\begin{frame}
\textbf{Créer un système de fichier avec mkfs}
\begin{itemize}
\item \textbf{mkfs} initialise un fichier système sur une nouvelle partition
\item ex: \texttt{mkfs -t ext3 -c /dev/hda2}
\item \textbf{t}: applique le système de fichier
\item \textbf{c}: vérifie les erreurs blocs sur le disque
\end{itemize}
\end{frame}

\begin{frame}
\textbf{Montage et démontage}
\begin{itemize}
\item \textbf{mount} monte un système de fichier et \textbf{umount} pour démonter
\item ex: \texttt{mount /dev/sdb3 /mnt/extra} (monte le périphérique
\texttt{/dev/sdb3} sur le point de montage \texttt{/mnt/extra})
\item ex: \texttt{mount -t vfat /dev/hdd1 /mnt/windows} (spécifie le point de
montage)
\item \texttt{man mount}
\item ex: \texttt{umount /dev/sdb3 ou umount /mnt/extra} pour démonter le périphérique
\item il est impossible de démonter un système de fichier \texttt{occupé}
\end{itemize}
\end{frame}

\begin{frame}
\textbf{Montage automatique}
\begin{itemize}
\item \textbf{/etc/fstab} contient les informations des systèmes de fichiers
contenu de l'administrateur
\item et qui sont montés au démarrage
\end{itemize}
\begin{tabular}{l l l l l l l}
\texttt{device} & \texttt{mount point} & \texttt{type} &
\texttt{options} & \texttt{dump} & \texttt{pass-no}\\
\texttt{/dev/hda3} & /home & /ext4 & defaults & 1 & 2
\end{tabular}
\end{frame}

\begin{frame}
\textbf{fstab}
\begin{itemize}
\item \textbf{device}: périphériques à monter
\item \textbf{mount point}: point de montage
\item \textbf{type}: système de fichier (ext2,vfat,auto,nfs)
\item \textbf{options}: noauto(non chargé au démarrage),ro(read-only),users
\item \textbf{dump}: \texttt{1} (système de fichier normale) ou \texttt{0}
(système de fichier removeable)
\item \textbf{pass-no}: \texttt{1} (système de fichier racine), \texttt{0}
système de fichier non monté au démarrage, \texttt{2} pour les autres systèmes
de fichier.
\end{itemize}
\end{frame}

\begin{frame}
\textbf{Maintenir un système de fichier}
\begin{itemize}
\item \textbf{df}: affiche le résumé des espaces libre sur la partition
\item \textbf{du}: affiche les informations de l'espace disque utilisé sur la
branche d'un répertoire
\item \textbf{fsck} est un utilitaire qui permet le contôle de l'intégrité des
partions
\item \texttt{ex: fsck /dev/sdb3} de préférence sur un système de fichier non
monté
\end{itemize}
\end{frame}


\section{Installation de logiciels}
\frame{\tableofcontents[current]}


\begin{frame}
Ajout/suppresion de logiciels depuis\
\textbf{les sources}
	\begin{block}{Compilation du programme}
	tar zxvf logiciel.tar.gz\\
	cd logiciel\\
	./configure\\
	make\\
	make install\\
	\end{block}
	\begin{block}{Avantages}
	Dernières versions\\
	Logiciel adapté à la plateforme, plus performant\\
	Options supplémentraire\\
	\end{block}
\end{frame}

\begin{frame}
	\begin{center}
	Ajout/suppresion de logiciels par\\
	\textbf{APT : Advanced Package Tool}
	\end{center}
	\begin{itemize}
	\item Dépots de paquets \texttt{/etc/apt/sources.list}
	\item \textit{install} : installation d'un paquet = téléchargement + installation + configuration 
	\item \textit{upgrade} : mise à jour de tous les paquets installés (suivi de sécurité)
	\item Notion de dépendances entre les paquets
	\item APT ne fait pas lui-même l'installation et la configuration, il passe par \textbf{dpkg}
	\end{itemize}
\end{frame}

\begin{frame}
	\begin{center}
	\texttt{aptitude} : le centre de commande APT
	\end{center}
	\begin{itemize}
	\item \textbf{/etc/apt/sources.list} : la configuration des dépôts
	\item \textbf{aptitude update} : lecture des paquets disponibles dans les dépots 
	\item \textbf{aptitude search} \textit{motclef}
	\item \textbf{aptitude show} \textit{paquet}
	\item \textbf{aptitude install} \textit{paquet}
	\item \textbf{aptitude safe-upgrade} et  \textbf{aptitude full-upgrade}
	\item et aussi : \textit{dpkg -l}
	\end{itemize}
\end{frame}

\begin{frame}
\begin{block}{ Les dépôts conseillés :}
	\begin{itemize}
	\item dépôts officiels, depuis le miroir le plus proche : \\
	 \texttt{deb \url{http://ftp.fr.debian.org/debian} squeeze main}
	\item dépôt de sécurité (obligatoire !) : \\
	 \texttt{deb http://security.debian.org/} squeeze/updates main
	\item dépot updates : \\
	\texttt{deb http://ftp.fr.debian.org/debian/ squeeze-updates main}
	\item dépôts backports (uniquement en cas de nécessité impérative) : \url{http://www.backports.org}
	\end{itemize}
\end{block}
NB : les autres dépots ne sont pas toujours dignes de confiance !
\end{frame}







%\section{Utilisation de l'editeur de fichiers}
%\frame{\tableofcontents[current]}


%\begin{frame}
%\textbf{Vi\texttt(Visual Editor)} 
%est un éditeur de texte, présent sur toutes les version de linux. \\
%Il existe le mode \textit{édition} et le mode \textit{commande} \\
%Pour ajout et suppression :
%\begin{itemize}
%    \item \textbf{insertion} : \texttt{Esc + i} 
%    \item \textbf{suppression} : caractère \texttt{Esc + x} ou d'une ligne \texttt{Esc + d + d}
%    \item \textbf{copier} : \textit{Esc + y + y} 
%    \item \textbf{coller} : \textit{Esc + p} 
%    \item \textbf{annuler} : \textit{u} 
%    \end{itemize}
%\end{frame}

%\begin{frame}
%    \begin{itemize}
%    \item \textbf{rechercher}: \textit{Esc + /text\_rechercher}
%    \item \textbf{substitue}:
%    \textit{ligne\_debut,ligne\_fins/chaine1/chaine2/g} sans confirmation \\
%    \textit{ligne\_debut,ligne\_fins/chaine1/chaine2/g} avec confirmation \\
%    \end{itemize}
%\end{frame}

%\begin{frame}
%    \begin{itemize}
%    \item \textbf{sauvegarder} : \textit{Esc} + \textit{:} + \textit{w} 
%    \item \textbf{sauvegarder et quitter} : \textit{Esc} + \textit{:} +
%    \textit{w} + \textit{q} 
%    \item \textbf{quitter sans sauvegarder} : \textit{Esc} + \textit{:} +
%    \textit{q} + \textit{!} 
%    \end{itemize}
%\end{frame}

\begin{frame}
\begin{center}
Travaux pratiques
\begin{itemize}
\item \textbf{vim}: l'editeur de texte
\item \textbf{rsync}: synchronisation
\item \textbf{tar}: archivage
\end{itemize}
\end{center}
\end{frame}

\section{Administrer son système}
\frame{\tableofcontents[current]}


\begin{frame}
\textbf{Utilisateur et groupe}
    \begin{itemize}
    \item tout ce qui utilise une machine linux est un \texttt{utilisateur}
    \item les utilisateurs ont diffèrents privilèges
    \item les utilisateurs peuvent appartenir à des groupes
    \item à chaque utilisateur son répertoire de base (ex: \texttt{/home/toto})
    \item seul l'utilisateur a les droits sur son répértoire
    \item chaque utilisateur est identifié par un \texttt{UID} (User IDentificator)
    \item \texttt{su} : permet de changer d'utilisateur
    \end{itemize}
\end{frame}

\begin{frame}
\textbf{Le Superutilisateur: Root}
    \begin{itemize}
    \item Chaque système Linux a un utilisateur \texttt{root}
    \item \texttt{root} est le 'tout-puissant'
    \item seul \textbf{root} peut \texttt{créer}, \texttt{supprimer},
    \texttt{modifier} un utilisateur
    \item le prompt du root se termine souvent par \texttt{\#}
    \item \texttt{su -} : permet de se connecter en tant que root
    \end{itemize}
\end{frame}

\begin{frame}
\textbf{Gestion des droits}
    \begin{itemize}
    \item \textbf{chown}: change l'appartenance du fichier ou du répertoire  à
    un utilisateur\\
    ex: \texttt{chown [toto] [logfile.txt]}
    \item \textbf{chgrp}: change l'appartenance du fichier ou du répertoire à
    un groupe \\
    ex: \texttt{chgrp [staff] [document.txt]}
    \item l'option \textbf{-R} permet de changer les droits de manière
    \texttt{récurssive} \\
    ex: \texttt{chown -R root /usr/local/share/music} \\
    ex: \texttt{chown -R toto.www-docs  /www/intranet/user/toto/}
    \end{itemize}
\end{frame}


\begin{frame}
\textbf{Permission sur les fichiers}
\begin{itemize}
\item Les permissions représentent les actions pouvant être affectées aux fichiers
\item \texttt{Trois} types de permissions pour les fichiers
\item \textbf{r} et \textbf{x} sont nécessaires à l'éxecution d'un script
\end{itemize}
\begin{tabular}{|l|l|l|}
   \hline
   \textbf{Permission} & \textbf{Lettre} & \textbf{Description} \\
   \hline
   Read & r & Permission de lire les données dans le fichier \\
   Write & w & Permission d'écrire, trunquer,\\ 
   & & et d'écraser les données dans le fichier \\
   Execute & x & Permission d'executer \\
   & & le contenu du fichier comme un programme\\
   \hline
\end{tabular}
\end{frame}

\begin{frame}
\textbf{Permission sur les répertoires}
    \begin{itemize}
    \item r,w,x ont aussi leur signification sur les répértoires
    \item Les permissions sur les répertoires sont un petit peu différents
\end{itemize}
\begin{tabular}{|l|l|l|}
    \hline
    \textbf{Permission} & \textbf{Lettre} & \textbf{Description} \\
    \hline
    Read & r & Permission d'avoir la liste du répertoire \\
    Write & w & Permission de créer, effacer,\\ 
    & & renommer les fichiers dans le répertoire \\
    Execute & x & Permission de changer le répertoire, ou  \\
    & & de l'utiliser comme répertoire intermédiaire \\
    \hline
\end{tabular}
\end{frame}

\begin{frame}
\textbf{Concept de bases}
\begin{itemize}
\item Un fichier a un \textbf{propriétaire} et un \textbf{groupe propriétaire}
\item Les permissions \texttt{r,w,x} sont spécifiés pour l'utilisateur, le
groupe d'utilisateur est le reste (les autres)
\end{itemize}
\end{frame}

\begin{frame}
\textbf{Examen des permissions}
\begin{itemize}
\item ls -l \\
drwxr-x--- 9 toto staff  4096 Oct 12 12:57 accounts \\
-rw-rw-r-- 1 toto staff 11170 Dec 9 14:11 report.txt
\item \texttt{d} répertoire, {-} fichier texte
\item 3 caractères rwx pour les permissions de \textbf{l'utilisateur}
propriétaire
\item \texttt{3 caractères rwx} pour les permissions du \textbf{groupe}
propriétaire
\item \texttt{3 caractères rwx} pour les permissions des \textbf{autres}
\end{itemize}
\texttt{Note: cp -p ou -a préserve les permissions et l'horodotage}
\end{frame}

\begin{frame}
\textbf{Changer les permissions des fichiers et des répertoires: chmod }
\begin{itemize}
\item les permissions peuvent être changées seulement par son \textbf{propriétaire} ou par le \textbf{superutilisateur}
\item \texttt{chmod a+x un-programme}
\item ajoute \textbf{(+)} les permissions d'éxecution \textbf{(x)} pour tous les
utilisateurs \textbf{(a)} sur le fichier \texttt{un-programme}
\end{itemize}
\end{frame}

\begin{frame}
\textbf{Spécification des permissions}
\begin{itemize}
\item les formats de spécifications:
\textbf{[ugoa] [+=-] [rwx]}
\item ajout de permissions: \texttt{u} pour l'utilisateur, \texttt{g} pour le
groupe, \texttt{o} pour les autres, \texttt{a} pour tout le monde.
\item \texttt{=} applique la permission au fichier
\item \texttt{+} ajoute la permission à ceux déjà appliquer
\item \texttt{-} retire les permissions
\item La dernière lettre indique laquelle des permissions utilisés \texttt{rwx}
\item ex: \texttt{chmod -R g+rwx,o+rx repertoire-public}
\end{itemize}
\end{frame}

\begin{frame}
\textbf{La méthode octale}
\begin{itemize}
    \item Les permissions peuvent être affecté grâce à des nombres
    \begin{tabular}{|l|l|l|l|}
       \hline
       & \textbf{Utilisateur} & \textbf{Groupe} & \textbf{Autres} \\
       \hline
       Lecture & 400 & 40 & 4 \\
       Ecriture & 200 & 20 & 2 \\
       Execution & 100 & 10 & 1 \\
       \hline
    \end{tabular}
    \item \texttt{chmod 664 *.txt}
    \item \texttt{chmod ug=rw,o=r *.txt}
    \end{itemize}
\end{frame}

%\begin{frame}
%\textbf{Permissions spéciales: Sticky}
%\begin{itemize}
%\item Un fichier ne peut être effacé que par son utilisateur dans un répertoire ayant une permission
%\texttt{sticky} 
%\item \texttt{ex:   ls -l -d /tmp}
%\texttt{drwxrwxrwt   30 root root 11264 Dec 21 09:35 /tmp}
%\item \texttt{chmod +t /data/tmp}
%\end{itemize}
%\end{frame}
%
%\begin{frame}
%\textbf{Permissions spéciales: Setgid}
%\begin{itemize}
%\item Si un répertoire est \textbf{'setgid'}(set group-id), les fichiers qui y
%sont crées acquiert les permissions du groupe de ce répertoire
%\item \texttt{ex:   ls -l -d /data/projects}
%\texttt{drwxrwsr-x 16 root staff 4096 Oct 19 13:14 /data/projects}
%\item \texttt{chmod g+s /data/projects}
%\end{itemize}
%\end{frame}
%


%\section{La sauvegarde}
%\frame{\tableofcontents[current]}

%\begin{frame}
%Pourquoi sauvegardé ?
%\begin{itemize}
%\item Fichier perdu, abimé
%\item Erreur humaine
%\item Crash disque
%\end{itemize}
%\end{frame}

%\begin{frame}
%\textbf{tar}
%\begin{itemize}
%\item Sauvegarder avec tar : \texttt{tar -cvf nom\_support liste\_fichiers}
%\item Consulter une archive : \texttt{tar -tvf  nom\_support}
%\item Restituer une archive : \texttt{tar -xvf nom\_support (liste\_fichiers)}
%\\
%Compression avec les options \textbf{z} (gzip) et \textbf{j} (bzip2) \\
%\end{itemize}
%end{frame}

%begin{frame}
%a commande \textbf{cpio} : sauvegarde sur la sortie standard
%begin{itemize}
%\item   Sauvegarder : \texttt{cpio -ov > nomsupport}\\
%\texttt{ex : find /home/oldaccounts | cpio -ov > /dev/support}
%\item Consulter une archive : \texttt{cpio -itv < nomsupport}
%\item Restituer une archive : \texttt{cpio -iv listefichiers < nomsupport}
%\end{itemize}
%\end{frame}

%\begin{frame}
%La commande \textbf{dd}: copie physique bloc à bloc
%\begin{itemize}
%\item \texttt{if= le fichier à copier}
%\item \texttt{of= fichier la copie}
%\item \texttt{bs= n}       taille des blocs
%\item \texttt{skip= n}     nombres de blocs sautés en entrée
%\item \texttt{seek= n}     nombres de bloc sautés en sortie
%\item \texttt{count= n}    nombre de bloc copiés
%\item \textit{exemple:} \texttt{dd if=/dev/hda1 of=/dev/sda1}
%\end{itemize}
%\end{frame}


%\begin{frame}
%\textbf{Partimage} : permet de faire l'image d'une partition
%\begin{itemize}
%\item Contrairement a dd, ne prend en compte que les bloc occupés
%\item Possibilité de compression (Gzip ou Bzip2)
%\item Fonctionnement réseau (NFS, mode client serveur)
%\item \textit{exemple:} \texttt{partimage -z1 -o -d save /dev/hda12
%/mnt/backup/redhat-6.2.partimg.gz}
%\end{itemize}
%\end{frame}

%\begin{frame}
%\textbf{Rsync}
%\begin{itemize}
%\item Outil open source permettant la sauvegarde incrémentale
%\item Sauvegarde de fichiers locaux
%\item Copie locale vers une machine distante et inversement
%\item Copie locale vers un serveurs rsync et inversement
%\item \textit{exemple:}{rsync -avF /src/path/ /dest/dir}
%\end{itemize}
%\end{frame}

%\begin{frame}
%Sauvegarde réseaux
%\begin{itemize}
%\item Avec \textbf{tar}\\
%\texttt{tar -zcf - /some/file | ssh <host> tar -zxf - -C /destination}
%\item Avec \textbf{partimage}\\
%\texttt{partimage -s <host> save /some/partition <file>}
%\item Avec \textbf{scp}\\
%\texttt{scp /some/file [\ more ... ]\ <host>:/destination/file}
%\item Avec \textbf{Rsync}\\
%\texttt{rsync -e ssh /some/file [\ more ... ]\
%<host>:/destination/file}

%\end{itemize}
%\end{frame}


\section{Gestion de processus} 
\frame{\tableofcontents[current]}

\begin{frame}
Gestion des appplications
\begin{itemize}
\item \textbf{Application} = ensemble de processus
\item Processus:\\
fichier exécutable / des bibliothèques partagées (.so) \\
un ou plusieurs “threads”
\item \textbf{Activation} = sh, init, inetd, cron
\item \textbf{Information} = fichiers de log
\end{itemize}
\end{frame}


\begin{frame}
\textbf{Les processus}
\begin{itemize}
\item Exécution d'une commande : création d'un processus
\item Attribution d'un PID unique
\item Différents états (actif, en attente, ...)
\item Associés à un compte utilisateur, à un ou plusieurs groupes
\item Création d'un processus via la primitive "fork", "exec"
\item Référence du père : PPID

\end{itemize}
\end{frame}

\begin{frame}
\textbf{L'environnement} \\
Le fils hérite de son père:
\begin{itemize}
\item le répertoire courant
\item le umask
\item le ulimit
\item variables d'environement
\item le répertoire racine (modification via « chroot »)
\end{itemize}
\end{frame}

\begin{frame}
\textbf{Gestion des processus}
\begin{itemize}
\item ps, top, pstree : lister les processus
\item kill, killall : envoyer des signaux a des processus
\item strace : lister les appels système
\item lsof : lister les fichiers ouverts
\end{itemize}
\end{frame}

\begin{frame}
\textbf{cron}
\begin{itemize}
\item Exécution périodique de commandes
\item Fichier /etc/crontab
\item Répertoire /etc/cron.* (hourly, daily, weekly, montly)
\end{itemize}
\texttt{17 * * * * root run-parts --report /etc/cron.hourly}
\end{frame}


\begin{frame}
\textbf{cron}
\begin{itemize}
\item Limitation des utilisateurs via les fichiers\\
/etc/cron.allow\\
/etc/cron.deny\\
(Si non présent : soit root soit tout le monde)\\
\item   Commande crontab : \\
\texttt{crontab fichier} soumission d'une table cron \\
\texttt{crontab -l}: visualisation des tables actives \\
\texttt{crontab -r}: arrêt de l'exécution de la table active \\
\texttt{crontab -e}: modification de la table active \\

\end{itemize}
\end{frame}

\begin{frame}
\textbf{Le syslog}
\begin{itemize}

\item Démon syslogd
\item Reçoit les messages d'erreur émis par le noyau / certains services
\item Permet de choisir la destination des log (fichiers, machine distante)
\item Différents critères : origine / sévérité
\item Fichier de configuration : \texttt{etc/syslog.conf}

\end{itemize}
\end{frame}

\begin{frame}
\begin{block}{/etc/syslog.conf}
\begin{tabular}{l l l}
user.notice  & & -/var/log/user.log \\
*.*       &    &  @log.exemple.com \\
daemon.*  &    &      -/var/log/syslog \\
lpr.!*   & &  /dev/xconsole \\ 
*.crit;kern.none   &     &    * \\
mail.info  & & root \\
*.=debug   &  & /dev/tty8   \\
\end{tabular}
\end{block}
\end{frame}

\begin{frame}
\begin{block}{Les sous systèmes}
\begin{tabular}{l l l}
authpriv   & &    service de sécurité et d'authentification\\
cron  &  &        service cron \\
daemon  &  &      les démons système\\
kern  &   &      noyau \\
syslog  &  &     lui-même\\
user  &   &      processus utilisateurs\\
lpr, ftp, mail, news, ...\\ & &
\end{tabular}
\end{block}
\end{frame}

\begin{frame}
\begin{block}{Les niveaux de sévérité}
\begin{tabular}{l l l}
emerg & & système inutilisable \\
alert & & intervention immédiate\\
crit & & erreur critique\\
err & & erreur\\
warn & & avertissement \\ 
notice & & événement normal \\
info & & information \\
debug & & test, mise au point \\
none & & ignorer les messages \\
\end{tabular}
\end{block}
\end{frame}

\begin{frame}
\begin{block}{Syntaxe des actions}
\begin{itemize}
\item ajouter le message à un fichier \texttt{(/var/log/message)}
\item envoyer le message à un serveur syslog \texttt{(@log.sn.auf)}
\item envoyer le message dans un tube nommé préexistant \texttt{(|/dev/xconsole)}
\item evoyer le message aux utilisateurs qui sont connecté \texttt{(root,roger)}
\item écrire sur la console texte \texttt{(/dev/tty8)}
\end{itemize}
\end{block}
\end{frame}



\section{Documentation} 
\frame{\tableofcontents[current]}

\begin{frame}
Sur le système, les pages de \texttt{man}
\begin{itemize}
    \item le format d'une page de manuel : \texttt{NAME}, \texttt{SYNOPSIS}, \texttt{DESCRIPTION},  \texttt{OPTIONS}, \texttt{FILES}, \texttt{SEE ALSO}, \texttt{BUGS}, ...
    \item \texttt{man man} : les différentes sections. En cas d'ambiguité, on précise : \texttt{man 1 crontab} ou \texttt{man 5 crontab}
    \item \texttt{whatis} et \texttt{apropos}
    \item en français : paquet \texttt{manpages-fr}
    \item le système \texttt{info}
\end{itemize}
\end{frame}

\begin{frame}
Sur le web : à boire et à manger !
    \begin{itemize}
    \item Les grands sites de documentations \textbf{validées} : \url{tldp.org}
    et \texttt{traduc.org}, \url{linux-france.fr}, \url{framabook.org}, 
    \item Les sites «constructeurs» : \url{www.debian.org},
    \url{help.ubuntu.com}, \url{fedoraproject.org}, \url{www.redhat.com}
    \item Les sites des logiciels : \url{www.apache.org}, \url{www.php.net}, ...
    \item Attention : aux sites \textit{wiki} publics, aux forums (par mail, sur le web ou sur les news), aux blogs et autres sites personels.
    \end{itemize}
\end{frame}


\begin{frame}
Au secours je suis perdu, aidez moi !
    \begin{itemize}
    \item Comment poser les questions de manière intelligente
    \item \url{http://www.linux-france.org/article/these/smart-questions/smart-questions-fr.html}
    \end{itemize}
\end{frame}



\end{document}
